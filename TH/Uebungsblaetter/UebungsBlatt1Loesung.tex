\documentclass[10pt]{scrartcl}

\usepackage[utf8]{inputenc}
\usepackage{tabularx}
\usepackage{longtable}
\usepackage[ngerman]{babel}
\usepackage[automark]{scrpage2}
\usepackage{amsmath,amssymb,amstext}
\usepackage[]{color}
\usepackage[]{enumerate}
\usepackage{graphicx}
\usepackage{polynom}
\usepackage{lastpage}
\usepackage[perpage,para,symbol*]{footmisc}
\usepackage{listings} 
\usepackage[pdfborder={0 0 0},colorlinks=false]{hyperref}
\usepackage[numbers,square]{natbib}
\usepackage{color}
\usepackage{colortbl}
\usepackage[absolute]{textpos}
\usepackage{float}
\usepackage[colorinlistoftodos,textsize=small,textwidth=2cm,shadow,bordercolor=black,backgroundcolor={red!100!green!33},linecolor=black]{todonotes}

\lstset{numbers=left, numberstyle=\tiny, numbersep=5pt, breaklines=true, showstringspaces=false} 
\restylefloat{figure}

%changehere
\def\titletext{Uebungsblatt 1}
\def\titletextshort{Praktikum 1}
\author{André Harms, Oliver Steenbuck}

\title{\titletext}

%changehere Datum der Übung
\date{19.04.2012}

\pagestyle{scrheadings}
%changehere
\ihead{TH1, Padberg}
\ifoot{Generiert am:\\ \today}

\cfoot{Oliver Steenbuck, André Harms}


\ohead[]{\titletextshort}
\ofoot[]{{\thepage} / \pageref{LastPage}}

\setlength{\parindent}{0.0in}
\setlength{\parskip}{0.1in}

\begin{document}
\maketitle

\setcounter{tocdepth}{3}
\tableofcontents

%	\listoftables                                 												% 
	\listoffigures  
	\lstlistoflistings	

\section{Aufgabe 1}
	\subsection{Formale Definition des Netzes}
	\begin{align}
	&N =\{P,T,W,M_0\}\\
	&P =\{p1,p2,p3,p4\}\\
	&T =\{t1,t2,t3\}\\
	&W(x,y) =\begin{cases}
			2 \text{ ;falls } (x,y) \in \{(t1,p2),(t2,p3)\} \\
			1 \text{ ;falls } (x,y) \in \{(p1,t1),(p2,t2), (p3,t3), (t3,p1), (t3,p4)\} \\
			0 \text{ ;sonst}
	     \end{cases}\\
	&M_0(x)=  \begin{cases}
			1 \text{ ;falls } x=p1\\
			0 \text{ ;sonst}
	     \end{cases}   
	\end{align}	

	\subsection{Schalthäufigkeit}
	Das Netz kann beliebig oft schalten.

\section{Aufgabe 2}
	\subsection{Formale Definition des Netzes}
	\begin{align}
	&N =\{P,T,W,M_0\}\\
	&P =\{p1,p2,p3,p4\}\\
	&T =\{t1,t2,t3\}\\
	&W(x,y) =\begin{cases}
			2 \text{ ;falls } (x,y) \in \{(t1,p2),(t2,p3)\} \\
			1 \text{ ;falls } (x,y) \in \{(p1,t1),(p2,t2), (p3,t3), (t3,p1), (t3,p4)\} \\
			0 \text{ ;sonst}
	     \end{cases}\\
	&M_0(x)=  \begin{cases}
			1 \text{ ;falls } x=p1\\
			0 \text{ ;sonst}
	     \end{cases}\\  
	&K(x)=  \begin{cases}
			7 \text{ ;falls } x=p1\\
			4 \text{ ;falls } x=p4\\
			\omega \text{ ;sonst}
	     \end{cases} 	      
	\end{align}	

	\subsection{Schalthäufigkeit}
	Nein, da durch die Kapazität auf $p4$ die Transition $t3$ maximal 4 mal geschaltet werden kann und $p1$ diese Transition benötigt. 

\section{Aufgabe 3}

\section{Aufgabe 4}

\end{document}

